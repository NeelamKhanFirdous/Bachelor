This chapter introduces the main topics needed to understand the development of this thesis.

%%%%%%%%%%%%%%%%%%%%%%%%%%%%%%%%%%%%%%%%%%%%%%%%%%%%%%%%%%%%%%%%%%%%%%%%%%%%%%%%%%%%%%%%%%%%%%
%%                                                                                          %%
%%                                Semantic Web Technologies                                 %%
%%                                                                                          %%
%%%%%%%%%%%%%%%%%%%%%%%%%%%%%%%%%%%%%%%%%%%%%%%%%%%%%%%%%%%%%%%%%%%%%%%%%%%%%%%%%%%%%%%%%%%%%%
There are huge amounts of text data on the internet. In fact, there is so much of it, that no human can possibly ever read and understand everything. That is why we try to use the computer to help us guide through the data. \\
There are several strategies for establishing order across texts, the most common being information retrieval (IR), information filtering (IF) and information extraction (IE)\cite{IE}. Information retrieval concerns itself with all the activities related to the organization of, processing of, and access to, information of all forms and formats. It can also be seen as a document retrieval system, since it is designed to retrieve information about the existence of documents relevant to a user query\cite{chowdhury2010IR}. Information filtering aims to remove irrelevant data from incoming streams of data items \cite{hanani2001IF}. Information Extraction refers to the automatic extraction of structured information such as entities or relations from unstructured sources\cite{sarawagi2008IE}. In contrast to IR systems, IE systems need to extract facts from the documents itself. The extracted data is often used for filling in databases, which are then available for various applications to further process the data\cite{klügl2014IE}. Since information is often spread across multiple sentences, understanding natural language is fundamental to IE\cite{IETaC}.