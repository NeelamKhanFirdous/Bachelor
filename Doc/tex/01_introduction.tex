%%%%%%%%%%%%%%%%%%%%%%%%%%%%%%%%%%%%%%%%%%%%%%%%%%%%%%%%%%%%%%%%%%%%%%%%%%%%%%%%%%%%%%%%%%%%%%
%%                                                                                          %%
%%                                    WHY                                   %%
%%                                                                                          %%
%%%%%%%%%%%%%%%%%%%%%%%%%%%%%%%%%%%%%%%%%%%%%%%%%%%%%%%%%%%%%%%%%%%%%%%%%%%%%%%%%%%%%%%%%%%%%%
The outbreak of the 2019 coronavirus disease (COVID-19) greatly impacted the entire world. Due to the virus being very contagious, hygiene became a lot more important and social interactions had to be reduced to a minimum. While the precise measures to reduce the spread of the virus differed from country to country, lockdowns were introduced in almost every part of the world. People were not allowed to go outside unless it was essential for their living. Companies were told to let the employees work from home, so they didn't need to leave the house. Clubs, bars and restaurants had to close down because of government orders as well as a lack of customers. Considering these and many other points, it is apparent that people were not driving around as much as before the pandemic. Rather they were staying at home, thus considerably reducing their transportation emissions. \\
Many scientists saw this as a unique opportunity to research how much of an impact such lockdowns have on air quality. While daily life was put on hold because of the restrictions, the streets of major cities were emptier than ever. Many articles regarding air quality have been published, however not all followed scientific methods and reviewing practices. In order to see all the data in one place, the research center Jülich accumulated every article pertaining to the change in air quality during lockdowns. The examined articles are all scientifically verified\cite{base}. These articles contain a great amount of data, which has to be processed manually. The research center Jülich processed each article individually and collected the air quality data. They set up a website which presents the results in various charts. \\
However, the process of manually having to check each research article proves to be very inefficient. Therefore, models for automated information extraction of such air quality data points could provide a fitting solution. Since scientists are still conducting new research regarding the connection between lockdowns and air quality, these models would also be useful for later database updates. By automatically extracting the data, the time spent skimming through the papers will be significantly reduced. On the contrary, the technology is errorprone. It is not expected to find every single data point that exists in the text. However, at the very least, it will provide a solid baseline for further examination. 

%%%%%%%%%%%%%%%%%%%%%%%%%%%%%%%%%%%%%%%%%%%%%%%%%%%%%%%%%%%%%%%%%%%%%%%%%%%%%%%%%%%%%%%%%%%%%%
%%                                                                                          %%
%%                                    Motivating Example                                    %%
%%                                                                                          %%
%%%%%%%%%%%%%%%%%%%%%%%%%%%%%%%%%%%%%%%%%%%%%%%%%%%%%%%%%%%%%%%%%%%%%%%%%%%%%%%%%%%%%%%%%%%%%%



%%%%%%%%%%%%%%%%%%%%%%%%%%%%%%%%%%%%%%%%%%%%%%%%%%%%%%%%%%%%%%%%%%%%%%%%%%%%%%%%%%%%%%%%%%%%%%
%%                                                                                          %%
%%                                      Contributions                                       %%
%%                                                                                          %%
%%%%%%%%%%%%%%%%%%%%%%%%%%%%%%%%%%%%%%%%%%%%%%%%%%%%%%%%%%%%%%%%%%%%%%%%%%%%%%%%%%%%%%%%%%%%%%
%%%%%%%%%%%%%%%%%%%%%%%%%%%%%%%%%%%%%%%%%%%%%%%%%%%%%%%%%%%%%%%%%%%%%%%%%%%%%%%%%%%%%%%%%%%%%%
%%                                                                                          %%
%%                                    Structure of the Book                                      %%
%%                                                                                          %%
%%%%%%%%%%%%%%%%%%%%%%%%%%%%%%%%%%%%%%%%%%%%%%%%%%%%%%%%%%%%%%%%%%%%%%%%%%%%%%%%%%%%%%%%%%%%%%

